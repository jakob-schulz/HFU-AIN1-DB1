\documentclass[a4paper,11pt,titlepage]{article}

\usepackage{ucs}
% per input encoding kann man Umlaute direkt einsetzten, aber  dann ist man von Font des jeweiligen Rechners abh"angig. Daher mag ich es nicht!
%\usepackage[utf8x]{inputenc}
\usepackage[german,ngerman]{babel}
\usepackage{fontenc}
\usepackage[pdftex]{graphicx}
%\usepackage{latexsym}
\usepackage[pdftex]{hyperref}
\usepackage{color}

\renewcommand*{\thesubsection}{\alph{subsection}.}

\begin{document}

% hier aktuelle Uebungsnummer einfuegen
\title{Datenbanken\\
Ausarbeitung Aufgabenblatt 3}

% Namen der Bearbeiter einfuegen

\author{Jakob Schulz}

% aktuelles Datum einfuegen

\date{23. Oktober 2023}

\maketitle{\thispagestyle{plain}}

\section{Aufgabe}
\subsection{}
Befehl: drop table \dots
\begin{verbatim}
create table orte 
(
ort_id int generated always as identity primary key,
name varchar(80),
land varchar(50),
einwohner int,
flaeche int
)
create table attraktionen(
a_id int generated always as identity primary key,
name varchar(50),
beschreibung varchar(50),
ort_id int,
foreign key(ort_id) references orte(ort_id)
)
\end{verbatim}
\subsection{}
\begin{verbatim}
insert into orte(name, land, einwohner, flaeche) 
values ('Köln', 'Deutschland', 1073096, 405010000);

insert into orte(name, land, einwohner, flaeche) 
values ('Stuttgart', 'Deutschland', 634830, 207400000);

insert into orte(name, land, einwohner, flaeche)
values ('München', 'Deutschland', 1472000, 310700000);

insert into orte(name, land, einwohner, flaeche) 
values ('Berlin', 'Deutschland', 3645000, 891800000);
		
insert into attraktionen(name, beschreibung, ort_id) 
values ('Brandenburger Tor', 
'Das B. T. ist ein frühklassizistisches Triumphtor', 4);

insert into attraktionen(name, beschreibung, ort_id) 
values ('Fernsehturm Berlin', 'Sendeturm mit schöner Aussicht', 4);

insert into attraktionen(name, beschreibung, ort_id)
values ('Hofbräuhaus München', 'Weltberühmtes Wirtshaus', 3);
\end{verbatim}
Versuch gegen referentielle Integrität zu Verstoßen: insert into attraktionen(name, beschreibung, ort\_id) values ('Europapark-Rust', 'Freizeitpark', 5);\\
Fehlermeldung: \\
 \textcolor{red}{Referentielle Integrität verletzt:"CONSTRAINT\_E8: PUBLIC.ATTRAKTIONEN FOREIGN KEY(ORT\_ID) REFERENCES PUBLIC.ORTE(ORT\_ID) (5)''}
\section{Aufgabe}
\subsection{}
Einf"ugen:\\
\begin{verbatim}
alter table orte add column laengengrad decimal(17,7);
alter table orte add column breitengrad decimal(17,7);
\end{verbatim}
\subsection{}
\begin{verbatim}
update orte set laengengrad = 6.9583 where ort_id = 1;
update orte set breitengrad = 50.9413 where ort_id = 1;
update orte set laengengrad = 9.1829 where ort_id = 2;
update orte set breitengrad = 48.7758 where ort_id = 2;
update orte set laengengrad = '11.5755' where ort_id = 3;
update orte set breitengrad = '48.1372' where ort_id = 3;
update orte set laengengrad = 13.3777 where ort_id = 4;
update orte set breitengrad = 52.5162 where ort_id = 4;
\end{verbatim}
\subsection{}
\begin{verbatim}
alter table attraktionen drop column beschreibung;
\end{verbatim}
\subsection{}
Tabellen löschen:
\begin{verbatim}
drop table attraktionen;
drop table orte;
\end{verbatim}
Tabellen wieder anlegen:
\begin{verbatim}
create table orte 
(
ort_id int generated always as identity primary key,
name varchar(80),
land varchar(50),
einwohner int,
flaeche int
);
create table attraktionen(
a_id int generated always as identity primary key,
name varchar(50),
beschreibung varchar(50),
ort_id int
);
alter table attraktionen 
add constraint ref_tabelle_orte foreign key(ort_id) references orte(ort_id);

\end{verbatim}
\section{Aufgabe}
\subsection{}
\begin{verbatim}
create table reise
(
id int generated always as identity primary key,
ziel varchar(20),
preis int,
dauer int,
verkehrsmittel varchar(20),
anfangsdatum varchar(20)
)
\end{verbatim}
\subsection{}
\begin{verbatim}
create table reise
(
id int generated always as identity primary key,
ziel varchar(20),
dauer int  check (dauer>0),
preis int  check (preis>= dauer* 50),
verkehrsmittel varchar(20),
anfangsdatum varchar(20)
)
\end{verbatim}
\subsection{}
\begin{verbatim}
create table reise
(
id int generated always as identity primary key,
ziel varchar(20),
dauer int  check (dauer>0),
preis int  check (preis>= dauer* 50),
verkehrsmittel varchar(20),
anfangsdatum varchar(20)
)
\end{verbatim}
\subsection{}
\begin{verbatim}
create table verkehrsmittel(
v_id int generated always as identity primary key,
name varchar(20)
);
create table ziel(
o_id int generated always as identity primary key,
name varchar(80)
);
create table reise(
r_id int generated always as identity primary key,
preis int,
dauer int,
anfangsdatum varchar(20),
o_id int references ziel,
v_id int references verkehrsmittel
)
\end{verbatim}
$\Rightarrow$Namen ausschreiben, hilfreich bei großen Tabellen\\
Daten Einfügen:
\begin{verbatim}
insert into ziel(name) values('Freiburg');
insert into ziel(name) values('München');
insert into ziel(name) values('Hamburg');
insert into verkehrsmittel(name) values('Auto');
insert into verkehrsmittel(name) values('Bahn');
insert into verkehrsmittel(name) values('Flugzeug');

insert into reise(preis, dauer, anfangsdatum, o_id, v_id)
values(200, 2, '10.10.2023', 1, 3);

insert into reise(preis, dauer, anfangsdatum, o_id, v_id)
values(1000, 8, '12.10.2023', 2, 2);

insert into reise(preis, dauer, anfangsdatum, o_id, v_id) 
values(150, 3, '9.8.2023', 3, 1);
\end{verbatim}
\section{Aufgabe}
\subsection{}
\begin{verbatim}
alter table titanic add column p_id int
generated always as identity primary key;
\end{verbatim}
\subsection{}
Befehl:
\begin{verbatim}
Select distinct embarked 
from titanic
\end{verbatim}
Werte: null, C, Q, S\\
\\
Befehl:
\begin{verbatim}
Select distinct sibsp
from titanic
\end{verbatim}
Werte:	0, 1, 2, 3, 4, 5, 8\\
\\
Befehl:
\begin{verbatim}
Select distinct parch
from titanic
\end{verbatim}
Werte:	0, 1, 2, 3, 4, 5, 6, 9
\subsection{}
Integritätsregeln:
\begin{verbatim}
alter table titanic add constraint titcanic_class 
check(class =3 or class = 2 or class = 1); 

alter table titanic add constraint titcanic_survived 
check(survived = 0 or survived = 1); 

alter table titanic add constraint titcanic_gender 
check(gender = 'male' or gender = 'female'); 

alter table titanic add constraint titcanic_embarked 
check(embarked = null or embarked = 'Q' or embarked ='C' or embarked = 'S');

alter table titanic add constraint titcanic_parch 
check(parch >= 0 and parch<= 6 or parch = 9);

alter table titanic add constraint titcanic_sibsp 
check(sibsp >=0 and sibsp <= 5 or sibsp = 8);

alter table titanic add constraint titcanic_fare 
check(fare >=0); 

alter table titanic add constraint titcanic_age
check(age >=0 or age = null); 

alter table titanic add constraint titanic_body 
check(body = null or body >0)
\end{verbatim}
$\Rightarrow$ Stellen sicher, dass für class, survived, gender, embarked, parch, sibsp, fare, age, und body keine falschen Eingaben getätigt werden können. \\
$\Rightarrow$null muss nicht gecheckt und wenn dann mit "is null".\\
\\
$\Rightarrow$ Alternativer und kürzerer Befehl:
\begin{verbatim}
alter table titanic add constraint titcanic_embarked 
check(embarked in 'Q', 'C', 'S');
\end{verbatim}
\subsection{}
\begin{verbatim}
create table embarked(
o_id int generated always as identity primary key,
ortkuerz varchar(5)
)
\end{verbatim}
\subsection{}
\begin{verbatim}
insert into embarked(ortkuerz)
Select distinct embarked
From titanic
\end{verbatim}
\subsection{}
\begin{verbatim}
alter table embarked add column ortsname varchar(30); 
\end{verbatim}
$\Rightarrow$ Ortsname hinzufügen
\begin{verbatim}
alter table embarked add constraint embarked_unique_kuerz unique(ortkuerz) 
\end{verbatim}
$\Rightarrow$ Stellt sicher, dass jedes kürzel nur einmal verwendet wird.\\
\\
Sofern man sich sicher ist, dass die Titanic nur in "`Cherbourg"', "`Queenstwon"' und "`Southampton"' angelegte, sind auch folgende Integritätsregeln möglich:
\begin{verbatim}
alter table embarked add constraint embarked_ortkuerz 
check(ortkuerz = 'C' or ortkuerz = 'Q' or ortkuerz = 'S') 
\end{verbatim}
$\Rightarrow$ erlaubt nur Ortskürzel mit den Buchstaben "`C"', "`Q"' oder "`S"'\\
\begin{verbatim}
alter table embarked add constraint embarked_ortsname 
check(ortsname = 'Cherbourg' or ortsname= 'Queenstown' or 
ortsname = 'Southampton') 
\end{verbatim}
$\Rightarrow$ erlaubt nur die Orte "`Cherbourg"', "`Queenstown"' oder "`Southampton"'
\subsection{}
\begin{verbatim}
update embarked
Set ortsname = 'Cherbourg'
Where ortkuerz = 'C'

update embarked
Set ortsname = 'Queenstown'
Where ortkuerz = 'Q'

update embarked
Set ortsname = 'Southampton'
Where ortkuerz = 'S'
\end{verbatim}

\end{document}
