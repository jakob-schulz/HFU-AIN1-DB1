\documentclass[a4paper, 11pt, titlepage]{article}
\usepackage{ucs}
\usepackage[german,ngerman]{babel}
\usepackage{fontenc}
\usepackage[pdftex]{graphicx}
\usepackage{xcolor}
\usepackage{listings}


\definecolor{dunkelblau}{RGB}{16, 55, 188}
\definecolor{orange}{RGB}{255, 60, 0}
\definecolor{gruen}{RGB}{18, 118, 34}
\definecolor{gelb}{RGB}{255, 200, 0}
\definecolor{lila}{RGB}{147, 18, 114}
\lstdefinestyle{sql}{
language=sql,
 literate=
    {'}{{\textquotesingle}}1, % ' wird als Apostroph ausgegeben
commentstyle=\color{gruen}, 
keywordstyle =\color{dunkelblau}, 
%Bis hier hin Farbgebung
frame=single, % Umrandung des Codes
rulecolor=\color{lightgray},
numbers=left, % Nummerierung hinzufügen (links)
numberstyle=\tiny, % Stil der Zeilennummern
stepnumber=1, % Schrittzahl für die Nummerierung
numbersep=5pt, % Abstand zwischen Nummerierung und Code
basicstyle=\sffamily, % Ändert die Schriftart des Codes
tabsize = 4, %Tab-Abstand
breaklines=true, %Zeilenumbruch
showstringspaces=false
}

\renewcommand*{\thesubsection}{\alph{subsection}.}

\begin{document}
\title{Datenbanken \\
Ausarbeitung \"Ubung 11}

\author{Jakob Schulz}

\date{\today}

\maketitle{\thispagestyle{plain}}
\noindent In den folgenden Aufgaben nutzen Sie die gleichen Daten wie in Aufgabenblatt 10. Ihre Lösungen 
dürfen nur syntaktische Elemente enthalten, die in der Vorlesung erläutert wurden. Insbesondere 
sind Konstruktionen wie limit oder Unterabfragen im from-Teil nicht zulässig.
\section{Aufgabe}
\begin{lstlisting}[style = sql]
Select count(*) as "Anzahl Passagiere"
From  titanic, embarked
where embarked.id = titanic.embarked_id
group by embarked.id;
\end{lstlisting}
\section{Aufgabe}
\subsection{Ermitteln Sie Alter und Namen des ältesten Passagiers. Wenn es mehrere älteste Passagiere 
gibt, müssen sie alle gefunden werden.}
\begin{lstlisting}[style = sql]
Select titanic.name, titanic.age
From titanic
Where titanic.age = 
(
Select max(age)
From titanic
);
\end{lstlisting}
Wieso nicht immer "`in"' verwenden?
\subsection{Anweisung ohne Join, sondern mit einer Unterabfrage formulieren}
\begin{lstlisting}[style = sql]
select count(*)
from titanic 
where titanic.embarked_id in 
(
Select embarked.id
from embarked
where embarked.city like '%town'
);
\end{lstlisting}
\subsection{Bonusaufgabe: Finden Sie eine Lösung für b., die den in-Operator nicht verwendet}
\begin{lstlisting}[style = sql]
Select count(*)
from titanic
where exists
(
	Select embarked.id
	from embarked
	where titanic.embarked_id = embarked.id and embarked.city like '%town'
)
\end{lstlisting}
Fragen, ob diese Lösung richtig ist!
\subsection{Welche id hat der Hafen, in dem die wenigsten Passagier zugestiegen sind? Formulieren Sie 
dazu eine einzige - verschachtelte - Abfrage}
\begin{lstlisting}[style = sql]
--ID des Hafens mit der geringsten Anzahl an zugestiegenen Passagieren wird ausgegeben:
Select embarked.id
From  titanic, embarked
where embarked.id = titanic.embarked_id	
group by embarked.id
having count(*) <= all	
(
	--man erhaelt Anzahl an Passageren, die in den Zustiegshaefen aufgenommen wurden:
	Select count(*)
	From  titanic, embarked
	where embarked.id = titanic.embarked_id
	group by embarked.id	
); 

\end{lstlisting}
Geht Abfrage auch ohne "`all"'?\\
Geht Abfrage auch einfacher?
\newpage
\subsection{Ermitteln Sie mit einer einzigen Anweisung die Namen der Passagiere, die in dem Hafen 
zugestiegen sind, in dem die wenigsten Passagier zugestiegen sind}
\begin{lstlisting}[style = sql]
--Name der Passagiere, die in dem Hafen zugestiegen sind werden ausgegeben
Select name
from titanic
where embarked_id in 
(
	--ID des Hafens mit der geringsten Anzahl an zugestiegenen Passagieren wird ausgegeben:
	Select embarked.id
	From  titanic, embarked
	where embarked.id = titanic.embarked_id
	group by embarked.id
	having count(*) <= all
	(
		--man erhaelt Anzahl an Passageren, die in den Zustiegshaefen aufgenommen wurden:
		Select count(*)
		From  titanic, embarked
		where embarked.id = titanic.embarked_id
		group by embarked.id	
	) 
);
\end{lstlisting}
\subsection{Problem der folgenden Abfrage ermitteln}
\begin{lstlisting}[style = sql]
select name, age 
from titanic 
where age>=all (select age from titanic)
\end{lstlisting}
Bei manchen Passagieren ist das Alter unbekannt ("`null"'). Da "`null"' hier nicht ignoriert wird, wird nichts ausgegeben. Zudem gilt f"ur kein Alter:\\
age $>$ null
\end{document}