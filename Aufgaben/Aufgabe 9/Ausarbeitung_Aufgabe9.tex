\documentclass[a4paper, 11pt, titlepage]{article}
\usepackage{ucs}
\usepackage[german,ngerman]{babel}
\usepackage{fontenc}
\usepackage[pdftex]{graphicx}
\usepackage{xcolor}
\usepackage{listings}


\definecolor{dunkelblau}{RGB}{16, 55, 188}
\definecolor{orange}{RGB}{255, 60, 0}
\definecolor{gruen}{RGB}{18, 118, 34}
\definecolor{gelb}{RGB}{255, 200, 0}
\definecolor{lila}{RGB}{147, 18, 114}
\lstdefinestyle{sql}{
language=sql,
 literate=
    {'}{{\textquotesingle}}1, % ' wird als Apostroph ausgegeben
commentstyle=\color{gruen}, 
keywordstyle =\color{dunkelblau}, 
%Bis hier hin Farbgebung
frame=single, % Umrandung des Codes
rulecolor=\color{lightgray},
numbers=left, % Nummerierung hinzufügen (links)
numberstyle=\tiny, % Stil der Zeilennummern
stepnumber=1, % Schrittzahl für die Nummerierung
numbersep=5pt, % Abstand zwischen Nummerierung und Code
basicstyle=\sffamily, % Ändert die Schriftart des Codes
tabsize = 4, %Tab-Abstand
breaklines=true, %Zeilenumbruch
showstringspaces=false
}

\renewcommand*{\thesubsection}{\alph{subsection}.}

\begin{document}
\title{Datenbanken \\
Ausarbeitung \"Ubung 9}

\author{Jakob Schulz}

\date{\today}

\maketitle{\thispagestyle{plain}}

\section{Aufgabe}
\subsection{DDL Anweisungen}
\begin{lstlisting}[style = sql]
create table fakultaet(
id int generated always as identity primary key,
name varchar(50)
);
\end{lstlisting}
\begin{lstlisting}[style = sql]
create table dozent(
id int generated always as identity primary key,
vorname varchar(50),
nachname varchar(50),
fakultaet int not null references fakultaet
);
\end{lstlisting}
\begin{lstlisting}[style = sql]
create table student(
id int generated always as identity primary key,
vorname varchar(50),
nachname varchar(50),
matrikelnummer int unique,
fakultaet int references fakultaet
);
\end{lstlisting}
\begin{lstlisting}[style = sql]
create table vorlesung(
id int generated always as identity primary key,
name varchar(80),
abkuerzung varchar(7) unique,
dozent int not null references dozent
);
\end{lstlisting}
\begin{lstlisting}[style = sql]
create table vorlesungStudent(
vorlesung int not null references vorlesung,
student int not null references student, 
primary key(vorlesung, student)
);
\end{lstlisting}
Mal ausprobieren, ob "`not null"'-Constraint hier weg gelassen werden kann!!
\newpage
\subsection{Daten einf"ugen}
\begin{lstlisting}[style = sql]
insert into fakultaet(name)
values('Informatik');
insert into fakultaet(name) 
values('Medical and Life Sciences');
insert into fakultaet(name) 
values('Digitale Medien');
insert into fakultaet(name) 
values('Industrial Technologies');
\end{lstlisting}
\begin{lstlisting}[style = sql]
insert into student(vorname, nachname, matrikelnummer) values('Jakob', 'Schulz', 23456);
insert into student(vorname, nachname, matrikelnummer) values('Julian', 'Faller', 12121);
insert into student(vorname, nachname, matrikelnummer) values('Leon', 'Kramar', 33345);
insert into student(vorname, nachname, matrikelnummer) values('Tim', 'Ringwald', 12345);
insert into student(vorname, nachname, matrikelnummer, fakultaet) values('Dominik', 'Mueller', 23496, 3);
\end{lstlisting}
\begin{lstlisting}[style = sql]
insert into dozent(vorname, nachname, fakultaet) values('Olaf', 'Neisse', 1);
insert into dozent(vorname, nachname, fakultaet) values('Ralf', 'Gerlich', 1);
insert into dozent(vorname, nachname, fakultaet) values('Pascal', 'Laube', 1);
\end{lstlisting}
\begin{lstlisting}[style = sql]
insert into vorlesung(name, dozent) 
values('Mathematik fuer Informatiker 1', 1);
insert into vorlesung(name, dozent) 
values('Einfuehrung in die Informatik', 2);
insert into vorlesung(name, dozent) 
values('Programmieren', 3);
\end{lstlisting}
\newpage
\begin{lstlisting}[style = sql]
insert into VORLESUNGSTUDENT values (4,1);
insert into VORLESUNGSTUDENT values (5,1);
insert into VORLESUNGSTUDENT values (6,1);
insert into VORLESUNGSTUDENT values (4,2);
insert into VORLESUNGSTUDENT values (5,2);
insert into VORLESUNGSTUDENT values (6,2);
insert into VORLESUNGSTUDENT values (4,3);
insert into VORLESUNGSTUDENT values (5,3);
insert into VORLESUNGSTUDENT values (6,3);
insert into VORLESUNGSTUDENT values (4,4);
insert into VORLESUNGSTUDENT values (5,4);
insert into VORLESUNGSTUDENT values (6,4);
\end{lstlisting}
\end{document}